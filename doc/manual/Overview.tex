
\abstract

This document describes the essential components of our V1.0 SCION prototype which
includes functional specifications of SCION network elements (e.g., servers,
routers, gateways), data structures of configuration files, and communication protocols among
these elements. In particular, this document focuses on the
specification of a working prototype and additional features that are not
described in the original paper. This is a working document which will be continuously updated
during the prototype creation process and includes discussions or comments on
current design issues. This document is written under the assumption that the original
SCION paper is read and understood in detail~\cite{ZHHCPA2011}.

\section{Introduction}

\subsection{SCION Overview}

SCION (Scalability, Control, and Isolation On Next-generation networks) is a
clean-slate Internet architecture designed to provide route control, failure
isolation, and explicit trust information for end-to-end communication. SCION
separates Autonomous Domains (ADs)\footnote{We use the term AD instead of AS to
  denote a smaller unit -- large domains with geographically dispersed units may
  constitute a single AS, however, we would consider them as multiple ADs.} into
groups of independent routing sub-planes, called isolation domains (IDs), which then
interconnect for global connectivity. Isolation domains provide natural isolation
for routing failures and human misconfiguration, give endpoints strong control
for both inbound and outbound traffic, provide meaningful and enforceable trust,
and enable scalable routing updates with high path freshness. As a result, SCION
provides strong resilience and security properties as an intrinsic consequence
of first-class security design principles, avoiding piecemeal add-on protocols
as security patches. Meanwhile, SCION only assumes that a few top-tier ISPs in
the isolation domain are trusted for providing reliable end-to-end communications,
thus achieving a small Trusted Computing Base (TCB). Both our security analysis
and evaluation results show that SCION naturally prevents numerous attacks and
provides a high level of resilience, scalability, control, and isolation. A
high-level SCION architecture is illustrated in Figure~\ref{fig:overview}.

\begin{figure*}[ht]
\centering
\includegraphics[width=.9\columnwidth]{./fig/overview.eps}
\caption{SCION Architecture.}\label{fig:overview}
\end{figure*}

We can summarize the design principles of SCION as follows.
\begin{enumerate}
\item {\bf Domain-based isolation.}
  \index{domain-based isolation}%
  \index{isolation!domain-based}%
  Dividing the routing control plane into
  independent domains. Isolation among independent domains protects routing in
  one domain from malicious activities and routing churn in other domains. This
  benefits both security and scalability while retaining reachability and path
  diversity across domains. For example, SCION enables frequent routing updates
  to periodically refresh path state, so that each AD always maintains a fresh
  (and accurate) network topology for efficient routing decisions. This
  separation also enables a separate root of trust for each domain, avoiding the
  conundrum of having to select a universally trusted single global root of
  trust. Finally, such isolation enables \textit{enforceable accountability},
  which enables punishment in case of misbehavior. In the current Internet,
  accountability is of limited utility as there is often no enforceability to
  handle detected misbehavior. By separating domains with enforceable
  accountability from domains without enforceability we can prioritize traffic
  accordingly and sidestep a major quagmire in the current Internet of how to
  prioritize traffic in case of DDoS attacks.
\item {\bf Mutually controllable path selection.} Joint path selection between
  source and destination. SCION greatly increases both the source and
  destination’s ability to affect, select and control the construction of the
  routes to and from themselves, while still respecting intermediate ISPs’
  routing policies.
\item {\bf Explicit trust and small TCB for end-to-end communication.} By
  segregating mutually distrustful entities into different isolation domains, each
  isolation domain can choose a coherent root of trust (e.g., a few tier-1 ISPs) for
  bootstrapping trust among ADs in the same isolation domain. As a result, an
  endpoint E knows and is able to choose explicitly whom to trust for achieving
  reliable end-to-end communication, while untrusted ADs in other isolation domains
  cannot affect the path discovery and route computation of E. Consequently, an
  entity only has to trust a small subset of the network thus achieving a small
  TCB for end-to-end communication.
\end{enumerate}


We briefly describe the end-to-end path construction in SCION and necessary functions for network elements. Furthermore, we introduce three servers, which help satisfy the design principles listed above.
\begin{itemize}
\item {\bf Path Construction. } A \ISDC starts propagating a Path Construction
  Beacon (PCB) to construct a path. PCB initiation is performed by {\bf Beacon
    Servers} in the \ISDC -- \BSs in the \ISDC generate a PCB
  periodically. For PCB propagation, \BSs in an \AD send PCBs to
  \BSs in neighboring domains -- \BSs in \TRAN \ADs
  propagate the PCB received from their parent \AD to their customer \ADs. A
  \BS adds its own \AD information (i.e., Opaque Field, digital
  signature, etc.) to the PCB. For signature verification, the \BS
  needs the public keys of the \AD who previously signed the PCB. The public key
  of each \AD is provided by the {\bf Certificate Server}, which stores the
  certificates of \ADs and provides \ADs' public keys to the \BS or
  the certificates to its customer \ADs when requested. When a border router
  receives a PCB, it forwards it to all \BSs within the \AD. Hence,
  for beacon propagation, border routers only need to know the \BSs
  within their own AD. The certificate server keeps track of all server
  locations (i.e., \BSs, certificate servers, and path servers as
  described below) and notifies to border routers of any topology change within
  the \AD.

\item {\bf Path Registration. } Once an \STUB \AD receives a PCB, it selects $k$
  paths to use and registers them with the {\bf Path Server} in the \ISDC. The \ISDC \PS
  processes path registration requests (e.g., whether to accept or reject) and
  stores this $k$ path registration information of the \STUB \AD.

\item {\bf End-to-end Path Resolution. }  A source \AD constructs a path to
  reach a destination \AD by composing its path to the \ISDC (up-path) with the
  path from the \ISDC to the destination \AD (down-path). To this end, when the
  \BS of an \AD receives a PCB, it provides the corresponding up-path (note that
  a PCB contains a path to the \ISDC) to {\em local} path servers. Local path
  servers store up-paths and provide them to endhosts. Meanwhile, down-paths are
  acquired from the \ISDC path server that holds $k$ down-paths to each \AD. The
  local path server sends path queries on behalf of endhosts and caches the
  results for later queries. The path server location is registered with the
  certificate server. A \BS extracts new paths to \ISDC from received PCBs and
  sends them to local path servers. Note that ADs can have multiple path servers
  for load balancing, as well as multiple path, certificate, and beacon servers
  for reliability.

\item {\bf Packet Forwarding. } Border routers distinguish two packet types:
  control and data packets. They forward control packets internally to the
  corresponding services; and forward data packets directly to egress interfaces
  (or hosts if the packet destination is within its \AD) specified in the packet
  header as an \textit{opaque field}. Since routers forward packets using the
  opaque fields in the packet header, they do not need to keep any forwarding
  table nor compute routes.
\end{itemize}

\subsubsection{\ISD: Root of Trust}

A \ISD is an independent {trust} network whose members reside in a common
contractual, legal, or regional environment. We envision \ISDs to grow to a
region where all internal \ADs can agree on a common root of trust, and where
accountability can be enforced on all internal \ADs. A \ISD consists of many \ADs
with various roles. \ISDC \ADs, which are generally big, national ISPs, provide
service to small ISPs (e.g., regional ISPs) or directly to \STUB \ADs (e.g.,
companies, universities). In principle, an \ISD needs at least one \ISDC \AD for
\ISD management and path construction. \TRAN \ADs provide network connectivity to
the \ISDC for customer \ADs and/or manage peering relationships with other \TRAN \ADs
(i.e., shortcuts) for improved service (in terms of path quality; e.g., path
length, path bandwidth). \STUB \ADs directly connect to endhosts and provide
domain resolution service (i.e., domain to path translation service) to those
hosts. Naturally, each \AD trusts its provider \AD(s), and as a consequence all
\ADs trust the \ISDC. Hence, all \STUB \ADs in the same \ISD trust each other by
trusting the \ISDC -- the Root of Trust -- without establishing individual trust
relationships. The Root of Trust is established by an agreement among \ISDC \ADs
and specified in the Root of Trust file (viz., Section~\ref{subsec:root-of-trust}).

\subsubsection{Autonomous Domain}
Each \AD provides the following services based on its type.
\begin{itemize}
\item {\bf \ISDC \AD: } \ISD membership service (both for \ISDC members and non-\ISDC members), certificate service (AD certificate issue/reference), PCB initiation, path record (i.e., handling path registration from \STUB \ADs), and domain resolution (i.e., domain to path translation).
\item {\bf \TRAN \AD: } certificate service (reference only), PCB propagation, path selection and registration (for endhosts), domain resolution.
\item {\bf \STUB \AD: } path selection and registration (for endhosts), domain resolution. 
\end{itemize}

\noindent{\bf Common services:} Autonomous domains contain servers for certificate management for internal entities (e.g., router, switch, server), up-path (to \ISDC) resolution service, topology management (external: border router to neighbor AD map, internal topology: server location), and key management (offline key, online key), 

\subsubsection{Servers}
To facilitate efficient routing, each \AD runs three servers. The required services at \ISDC \AD, \TRAN \AD, and \STUB \AD are annotated by \CAD, \TAD, and \EAD, respectively. Internal services and external services indicate those services provided within an \AD and across \ADs, respectively.
%\begin{itemize}

\noindent{\bf Certificate Server: } 
\begin{itemize}
\item \AD certificate service: respond to certificate requests (\CAD, \TAD,\EAD) 
\item \AD certificate management: issue certificate(\CAD), store certificates (\CAD,\TAD,\EAD)
\item Entity certificate service/management: certificates of internal entities (\CAD,\TAD,\EAD)  
\item Key management: manage the secret key for internal secure communication (\CAD,\TAD,\EAD), generate/distribute opaque field generation (OFG) key (\CAD,\TAD,\EAD) 
\item Topology management: distribute internal server locations to border routers (all external messages are forwarded to the corresponding servers by border routers, without revealing server locations outside the \AD.) (\CAD,\TAD,\EAD), manage database of border router (interface) to \AD mapping
\end{itemize}

\noindent{\bf Beacon Server: }
\begin{itemize}
\item PCB generation: initiate a PCB for path construction (\CAD)
\item PCB propagation: gather all PCBs arrived from provider \ADs, select $k$ PCBs (for every PCB propagation period), propagate selected PCBs to customer \ADs (\TAD)
\item path selection/registration: select $k$ down-paths from \ISDC and register them to the \ISDC path server (\TAD,\EAD)
\item path distribution: distribute up-paths to path servers (\TAD,\EAD) 
\end{itemize}

\noindent{\bf Path Server: }
\begin{itemize}
\item path record: keep $k$ down-paths for each \AD within an \ISD (\TAD)
\item domain resolution: translate the identifier of a destination \AD to the down-path(s) (i.e., from the \ISDC to the \AD) (\CAD,\TAD,\EAD) (\TAD,\EAD resolve a path either by sending a query to the \ISDC path server or using cached results) 
\item path caching: cache down-paths to destination \ADs and up-paths to \ISDC (\TAD,\EAD)
\end{itemize}


\subsection{Implementation Principles}
The SCION architecture and implementation are guided by the following principles:

\begin{enumerate}
\item {\bf Simple Router. }
Routers run with very simple functions: forwarding without per-flow and per-destination state, and efficient path verification.
%SCION edge routers forward all PCBs to the Beacon Server
%No huge routing table and forwarding table

%No complex route selection policies

\item {\bf Simple Configuration. }
An \AD should be able to add and configure routers easily. Complex configuration
can lead to outages or limited availability.

%Border routers need to know ``Keys'' and ``Server location''

\item {\bf Efficient Forwarding. }
The basic function (i.e., packet forwarding) of a router should be fast enough to operate at a line speed. It is desirable for routers to forward packets in $O(1)$ time (i.e., to remove any dependency on other factors such as the forwarding table size). 
%No FIB lookup
%A single hash operation

\item {\bf Scalability. }
Path advertisement and construction should be scalable to easily accommodate ever growing network size. For scalability, it is desirable to minimize any state that need to be kept at individual routers (or servers).

\item {\bf Rebootability. } The Internet should be able to reach a globally
  stable state when some or all network elements restart. In essence,
  rebootability means that no circular dependencies exist that would lead to a
  deadlock when a network is ``rebooted.''

\item {\bf Clean Trust Model. } The root of trust should be easily set up and
  the flow of trust should be {\em explicit}, such that the TCB for each network
  operation can be simply enumerated.

\end{enumerate}

%Outcomes: Highly resilient Internet architecture that can be incrementally deployed besides the current IP protocols. SCION will provide strong availability and reliability guarantees for a variety of traffic classes. 

%No global route advertisement

%No routing table management

%\end{itemize}

%\subsection{Bootstrapping}
